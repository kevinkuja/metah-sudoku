\section{Conclusiones}

En este trabajo, hemos presentado, a nuestro entender, dos aplicaciones de metaheurísticas para resolver el  popular rompecabezas Sudoku. Hemos aplicado los algoritmos a distintos problemas tomados de sitios (de distintos grados de "dificultad"), y hemos visto en todas nuestras pruebas de que los algoritmos no llegan a conseguir cierta optimalidad (como suele ser el caso de las técnicas de optimización), logrando mejores resultados con el Simulated annealing respecto a la Colonia de Hormigas, pero consistentemente encuentran la solución en un tiempo razonable, ya que estamos hablando de un problema NP Completo. Para tener éxito, no depende necesariamente de casos problemáticos sino de la logica que se aplica en encontrar la solución (y de las distintas técnias propias del Sudoku que se puedan aplicar).\\
Por último, aunque hemos demostrado en este informe que este tipo de enfoque de búsqueda estocástico es capaz de resolver una gran variedad de diferentes instancias sin ayuda, tal vez el punto más saliente que surge de esta investigación en su conjunto es el evidente potencial de la combinación de estas dos técnicas con una optimización propia basado en experimentos para corregir y mejorar los desempeños.  Por lo tanto, un algoritmo híbrido para sudoku (y otros problemas relacionados) que, por ejemplo, toma una instancia de problema y sigue el metodología de  llenar tantas celdas como sea posible a través de reglas lógicas (mejor explicado en la sección de la construcción de la Solución Inicial), y después de conmutación de soluciones con una técnica de búsqueda estocástica. Esto tiene el potencial de proporcionar un algoritmo final mucho más potente que cualquiera de estas dos técnicas de forma individual. 

