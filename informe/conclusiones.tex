\section{Conclusiones}
A modo de conclusión, en base a los experimentos y resultados obtenidos, tenemos varios puntos para destacar:\\
\begin{enumerate}
  \item 
\end{enumerate}

En este trabajo, hemos presentado, a nuestro entender, dos aplicaciones de metaheurísticas para resolver el  popular rompecabezas Sudoku. Hemos aplicado los algoritmos a distintos problemas tomados de sitios (de distintos grados de "dificultad"), y hemos visto en todas nuestras pruebas de que los algoritmos no llegan a conseguir cierta optimalidad (como suele ser el caso de las técnicas de optimización), logrando mejores resultados con el Simulated annealing respecto a la Colonia de Hormigas, pero consistentemente encuentran la solución en un tiempo razonable, ya que estamos hablando de un problema NP Completo. Para tener éxito, no depende necesariamente de casos problemáticos siendo lógica solucionable.
