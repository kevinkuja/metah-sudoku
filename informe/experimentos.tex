\section{Experimentos}

En primera instancia realizamos experimentos con ambas metaheurísticas a sudokus de distintos niveles de dificultad [3] y randoms (estos son sudokus solucionados a los que les blanqueamos de manera random 30 casillas). El objetivo es comparar los algoritmos entre sí para ver cuan efectivos son.


\begin{table}[ht]
\centering
\begin{tabular}{|l|l|l|}
\hline
          & \textbf{Simulated Annealing} & \textbf{Colonia de Hormigas} \\ \hline
{Escenarios probados} &       115                       &        115                      \\ \hline
{Total Solucionados} &                      70 (60\%)      &              60 (52\%)               \\ \hline
{Dificultad baja} &                     71 \%         &              69\%                \\ \hline
{Dificultad moderada} &                  40 \%            &             25 \%                 \\ \hline
{Dificultad alta} &                         16\%     &            5 \%                  \\ \hline
{Random} &                         63\%     &            60 \%                  \\ \hline
\end{tabular}
\end{table}



\subsection{Simulate annealing}
Para los experimentos elegimos un factor de enfriamiento en 0.5 y una temperatura inicial de 1. La justificación es en base a los experimientos realizados y que mostraremos en esta sección. \\ \\

El siguiente grafico es el resultado del tiempo que toma solucionar un Sudoku en donde partiendo de una solución existente blanqueamos celdas random y observamos como lo soluciona el algoritmo:\\
\includegraphics[scale=0.5]{imgs/randomSA.png}	\\
Clarmente se puede observar que a medida que el tablero esta mas lleno el tiempo es menor, eso es debido a la suma de varios factores, como que inicialmente infiere más celdas, son menos las vacias (que se convierte en menos iteraciones) y menos búsqueda de soluciones vecinas.\\\\

A continuación corrimos el algoritmo con 25 soluciones de diferentes niveles de dificultad, variando el factor de enfriamiento (o alfa):\\
\includegraphics[scale=0.7]{imgs/porc_soluc.png}	\\
Si bien con valores intermedios (aprox 0.5) es donde se concentran las mejores soluciones, y que la mayor cantidad se logra con un 0.1, creemos en base a anteriores pruebas que el el factor de enfriamiento debería ser siempre mayor de 0.5 para que sea algo "lento" y no siempre elija ir por soluciones vecinas pero que tampoco las restrinja totalmente. 

\subsection{Hormigas}

Uno de los experimentos que realizamos es el tiempo de computo que demanda este algoritmo para finalizar, mas alla de sí efectivamente logra solucionarlo o no. Como mencionamos anteriormente utilizamos un probabilidad de 2 para el seguimiento de la feromona, en el siguiente experimento veremos el porque de esta eleccion.

\includegraphics[scale=0.5]{imgs/resultados_random_hormigas.png}	\\
En este caso también podemos observar que a medida que el tablero esta mas lleno el tiempo es mucho menor. A partir de las 35 celdas blanqueadas aproximadamente vemos que el tiempo no es del todo regular, esto es porque hay casos en los que encuentra la solución rapidamente devido a su comportamiento random. Por otro lado los casos en los que no encuentra solución son los que mas demoran porque todas las hormigas intentan encontrarla, a diferencia de cuando se encuentra una solución ya que ahí frena la heurística y no siguen probando las demás hormigas.


\includegraphics[scale=0.5]{imgs/solucion_hormigas_proba.png}	